\documentclass[]{article}

\usepackage[left=2.00cm, right=2.00cm, top=2.00cm, bottom=2.00cm]{geometry}
\usepackage[spanish,es-noshorthands]{babel}
\usepackage[utf8]{inputenc} % para tildes y ñ

%opening
\title{Práctica 4. Exploración de grafos}
\author{Luis José Quintana Bolaño \\ % mantenga las dos barras al final de la línea y este comentario
luisjquintana@gmail.com \\ % mantenga las dos barras al final de la línea y este comentario
Teléfono: 956535843 \\ % mantenga las dos barras al final de la linea y este comentario
NIF: 49073584w \\ % mantenga las dos barras al final de la línea y este comentario
}


\begin{document}

\maketitle

%\begin{abstract}
%\end{abstract}

% Ejemplo de ecuación a trozos
%
%$f(i,j)=\left\{ 
%  \begin{array}{lcr}
%      i + j & si & i < j \\ % caso 1
%      i + 7 & si & i = 1 \\ % caso 2
%      2 & si & i \geq j     % caso 3
%  \end{array}
%\right.$

\begin{enumerate}
\item Comente el funcionamiento del algoritmo y describa las estructuras necesarias para llevar a cabo su implementación.

Para todos los casos se ha usado la misma estructura:
\begin{lstlisting}
typedef std::vector <std::pair <std::pair<int, int>, float> > values_t;
\end{lstlisting}

\item Incluya a continuación el código fuente relevante del algoritmo.

La tabla de subproblemas resueltos es representada mediante una matriz con columnas del 0 al máximo número de ases disponibles, y un número de filas igual al de defensas disponibles.

La posición de la fila equivale al indice de cada defensa en la lista, y al del valor de esta en el array de valores previamente calculado.



\end{enumerate}

Todo el material incluido en esta memoria y en los ficheros asociados es de mi autoría o ha sido facilitado por los profesores de la asignatura. Haciendo entrega de esta práctica confirmo que he leído la normativa de la asignatura, incluido el punto que respecta al uso de material no original.

\end{document}

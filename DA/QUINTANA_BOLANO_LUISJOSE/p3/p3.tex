\documentclass[]{article}

\usepackage[left=2.00cm, right=2.00cm, top=2.00cm, bottom=2.00cm]{geometry}
\usepackage[spanish,es-noshorthands]{babel}
\usepackage[utf8]{inputenc} % para tildes y ñ
\usepackage{graphicx} % para las figuras
\usepackage{xcolor}
\usepackage{listings} % para el código fuente en c++

\lstdefinestyle{customc}{
  belowcaptionskip=1\baselineskip,
  breaklines=true,
  frame=single,
  xleftmargin=\parindent,
  language=C++,
  showstringspaces=false,
  basicstyle=\footnotesize\ttfamily,
  keywordstyle=\bfseries\color{green!40!black},
  commentstyle=\itshape\color{gray!40!gray},
  identifierstyle=\color{black},
  stringstyle=\color{orange},
}
\lstset{style=customc}


%opening
\title{Práctica 3. Divide y vencerás}
\author{Luis José Quintana Bolaño \\ % mantenga las dos barras al final de la línea y este comentario
luisjquintana@gmail.com \\ % mantenga las dos barras al final de la línea y este comentario
Teléfono: 956535843 \\ % mantenga las dos barras al final de la linea y este comentario
NIF: 49073584w \\ % mantenga las dos barras al final de la línea y este comentario
}


\begin{document}

\maketitle

%\begin{abstract}
%\end{abstract}

% Ejemplo de ecuación a trozos
%
%$f(i,j)=\left\{ 
%  \begin{array}{lcr}
%      i + j & si & i < j \\ % caso 1
%      i + 7 & si & i = 1 \\ % caso 2
%      2 & si & i \geq j     % caso 3
%  \end{array}
%\right.$

\begin{enumerate}
\item Describa las estructuras de datos utilizados en cada caso para la representación del terreno de batalla. 

Para todos los casos se ha usado la misma estructura:
\begin{lstlisting}
typedef std::vector <std::pair <std::pair<int, int>, float> > values_t;
\end{lstlisting}

\item Implemente su propia versión del algoritmo de ordenación por fusión. Muestre a continuación el código fuente relevante. 

La tabla de subproblemas resueltos es representada mediante una matriz con columnas del 0 al máximo número de ases disponibles, y un número de filas igual al de defensas disponibles.

La posición de la fila equivale al indice de cada defensa en la lista, y al del valor de esta en el array de valores previamente calculado.



\item Implemente su propia versión del algoritmo de ordenación rápida. Muestre a continuación el código fuente relevante. 

\begin{lstlisting}
// sustituya este codigo por su respuesta
void selectDefenses(...) {

    unsigned int cost = 0;
    std::list<Defense*>::iterator it = defenses.begin();
    while(it != defenses.end()) {
        if(cost + (*it)->cost <= ases) {
            selectedIDs.push_back((*it)->id);
            cost += (*it)->cost;
        }
        ++it;
    }
}
\end{lstlisting}

\item Realice pruebas de caja negra para asegurar el correcto funcionamiento de los algoritmos de ordenación implementados en los ejercicios anteriores. Detalle a continuación el código relevante.

El algoritmo es categorizado como algoritmo voraz, puesto que busca un resultado local aplicando una función de factibilidad y descarta los elementos ya comprobados (en este caso, las posiciones).


\item Comente y justifique a continuación los resultados esperados en cada caso. Suponga un terreno de batalla cuadrado en todos los casos. 

Tomando un tablero de $n$ casillas y $m$ defensas:

Para el caso sin preodenación, se realizarán un máximo de $\displaystyle\sum_{i=0}^{m} n-i$, ya que tras cada acceso se elimina de la estructura que contiene las casillas el elemento comprobado.

Tanto en el caso preordenado por fusión como en el de ordenacióin rápida, el algoritmo es de orden $n \log n$

Pra el montículo, monte y culo.

\item Incluya a continuación una gráfica con los resultados obtenidos. Utilice un esquema indirecto de medida (considere un error absoluto de valor 0.001 y un error relativo de valor 0.001). Considere en su análisis los planetas con códigos 1500, 2500, 3500,..., 10500. Incluya en el análisis los planetas que considere oportunos para mostrar información relevante.

\begin{lstlisting}

std::vector <std::pair <std::pair<int, int>, float> > evaluateDefenses (int nCellsWidth, int nCellsHeight, float cellWidth, float cellHeight, Defense* mainTower){

	std::vector <std::pair <std::pair<int, int>, float> > cellVal (nCellsWidth*nCellsHeight);
	float maxDistance = euclideanDistance(0,(nCellsWidth-1)*cellWidth,0,(nCellsHeight-1)*cellHeight);

	int i=0;
	for (int x=0; x<nCellsWidth; ++x){
		for (int y=0; y<nCellsHeight;++y){
			//First values are assigned according to proximity to the main tower
			cellVal[i++]=std::make_pair(std::make_pair(x,y),(maxDistance-euclideanDistance(x*cellWidth,mainTower->position.x,y*cellHeight,mainTower->position.y))/maxDistance);
		}
	}

	return cellVal;
}


//DEFENSES================================================================================
	//Cell values for the position of the defensive towers are calculated
	//A higher value represents a better position.
	cellVal = evaluateDefenses (nCellsWidth,nCellsHeight,cellWidth,cellHeight,*currentDefense);

	//The values get sorted descendingly to try and place the towers in the best positions
	std::sort(cellVal.begin(), cellVal.end(), comparePair);

	//We try every position from highest value to lowest
	//Each tower is placed when a feasible position is find
	currentCell = cellVal.begin();
	while(currentCell != cellVal.end() && currentDefense != defenses.end()){
		if(feasibility(cellWidth,cellHeight,currentCell->first.first,currentCell->first.second,currentDefense,obstacles,defenses)){
			(*currentDefense)->position.x = currentCell->first.first * cellWidth + cellWidth * 0.5f;
        	(*currentDefense)->position.y = currentCell->first.second * cellHeight + cellHeight * 0.5f;
        	(*currentDefense)->position.z = 0;
        	++currentDefense;
		}
		++currentCell;
	}
\end{lstlisting}


\end{enumerate}

Todo el material incluido en esta memoria y en los ficheros asociados es de mi autoría o ha sido facilitado por los profesores de la asignatura. Haciendo entrega de este documento confirmo que he leído la normativa de la asignatura, incluido el punto que respecta al uso de material no original.

\end{document}
